\documentclass[11pt]{article}
\usepackage[utf8]{inputenc}
\usepackage[T1]{fontenc}
\usepackage{ctex}
\usepackage{float}
\usepackage{graphicx}
\usepackage{longtable}
\usepackage{wrapfig}
\usepackage{rotating}
\usepackage[normalem]{ulem}
\usepackage{amsmath}
\usepackage{amssymb}
\usepackage{siunitx}
\usepackage{capt-of}
\usepackage{imakeidx}
\usepackage{hyperref}
\hypersetup{
    colorlinks=true,
    citecolor=black,
    linkcolor=blue,
    filecolor=magenta,
    urlcolor=cyan,
    pdftitle={Overleaf Example},
    pdfpagemode=FullScreen,
    }
\usepackage{xcolor}
\usepackage{listings}
\definecolor{codegreen}{rgb}{0,0.6,0}
\definecolor{codegray}{rgb}{0.5,0.5,0.5}
\definecolor{codepurple}{rgb}{0.58,0,0.82}
\definecolor{backcolour}{rgb}{0.95,0.95,0.92}
\lstdefinestyle{mystyle}{
    backgroundcolor=\color{backcolour},
    commentstyle=\color{codegreen},
    keywordstyle=\color{magenta},
    numberstyle=\tiny\color{codegray},
    stringstyle=\color{codepurple},
    basicstyle=\ttfamily\footnotesize,
    breakatwhitespace=false,
    breaklines=true,
    captionpos=b,
    keepspaces=true,
    numbers=left,
    numbersep=5pt,
    showspaces=false,
    showstringspaces=false,
    showtabs=false,
    tabsize=2
}
\lstset{style=mystyle}
\usepackage{footmisc}
\usepackage{biblatex}
\usepackage[most]{tcolorbox}
\newtcolorbox{notebox}[0]{colback=yellow!5!white,
colframe=yellow!75!black,fonttitle=\bfseries,parbox=false,breakable,
title={Note}}
\newtcolorbox[auto counter, number within=section, list inside=question]{qbox}[1]{colback=blue!5!white,
colframe=blue!75!black,fonttitle=\bfseries,parbox=false,breakable,
title={小问题~\thetcbcounter:#1}}
\newtcolorbox[auto counter, number within=section, list inside=src]{readsrcbox}[1]{colback=green!5!white,
colframe=green!75!black,fonttitle=\bfseries,parbox=false,breakable,
title={读源码~\thetcbcounter:#1}}
\usepackage{csquotes}

\addbibresource{bib.bib}
\date{\today}
\title{操作系统案例分析}
\makeindex[intoc, columnseprule, options= -s style.ist]
\begin{document}
\lstset{
	breaklines=true,
	basicstyle=\ttfamily}
\title{OS Case Study}

\maketitle
\tableofcontents

\newpage

\section{历史}
\subsection{Linux 内核的历史}
Linux内核\index{Linux}最初是1991年,当时的赫尔辛基大学本科生Linus Torvalds出于兴趣为自己的386计算机编写的,开发使用的操作系统是教学用的Minix、编译器为GNU的GCC.
当时的背景是这样的:自由软件运动的GNU项目运转了近十年,且取得了很大的进展,GCC作为自由的编译器已经相当成功,其他用户态的工具都比较完善,但是内核项目仍处于早期阶段.
在此之前,Unix于1970年发布,它的成功使得学界和商业界都乐于借鉴其模式和思想.
因此作为一本操作系统教材的配套实现,Minix部分使用了Unix的理念,而GNU也把开发出与Unix兼容的自由操作系统为目标.
Linux 0.1版本发布后,世界上许多开发者对Linus的操作系统很感兴趣,他们加入了Linux的内核的开发,并将一些GNU系统的组件移植到Linux内核,最终使之成为GNU系统事实上的内核.

1992年Linus将Linux 0.99 在GNU General Public License 下发布,Linux正式成为自由软件.

2000年左右,在以著名的论文The Cathedral and the Bazaar为代表的开源运动的影响下,Linux作为开源软件的开发模式开始得到许多公司的重视.
许多硬件和软件公司在这时都开始为Linux提供支持,并加入Linux内核的开发.

如今(2020s)绝大多数互联网服务器运行Linux 发行版\cite{OSUsageT34:online},Linux内核也广泛用于嵌入式设备和智能手机.
作为成熟的可移植的自由类Unix内核,Linux既成为了开源运动的成功范例又为自由软件运动作出了突出的贡献.
\subsection{发行版Arch Linux}
Arch Linux \index{Arch Linux}是程序员兼吉他手Judd Vinet在2002年创建的一个滚动发行的Linux发行版.
其主要特点在于拥有可以自动管理依赖的包管理器pacman和受FreeBSD启发的包构建系统\cite{DistroWa81:online}.
由于Arch Linux上构建和分享软件包非常简单,Arch Linux用户可以使用的软件包非常丰富.
其简易性和以用户为中心\cite{ArchLinux15:online} 的优点使其成为滚动发行模式的代表和最受桌面用户欢迎的Linux发行版之一.

Arch Linux用户可以选择和配置自己的Linux内核,一般使用的是最新的stable分支上的普通的Linux内核.
\begin{notebox}
	由于发行版的内核可以自己配置,发行版的区别对于本案例分析的影响不大,后文的分析将不太涉及发行版之间的区别.
\end{notebox}

\section{Linux内核的架构}
Linux是宏内核,即整个内核是一个共用地址空间的二进制程序.
宏内核架构是相对于微内核而言的.
微内核架构基于消息传递,作为内核的二进制程序只有最基本的功能,其他的不同组件处于不同的地址空间,它们之间通过消息传递的方式来进行交互.
Linux内核采用宏内核的架构主要是出于性能方面的考虑.\cite{silberschatz2021operating}
同一地址空间的各个部分可以直接相互调用,没有消息传递的开销.

宏内核的架构不代表Linux内核不能采用模块化的设计,这得益于动态链接技术.
Linux内核由常驻内存的内核镜像\lstinline{vmlinux}和动态加载的各种模块构成.
使用运行时加载的模块的好处有很多.
首先,这些模块在加载后就处于核心态,可以调用内核中的任何代码,访问硬件也没有限制,这可以使内核的设计者适度地划分各个功能之间的界限——避免功能之间耦合度过高的同时也不必在不同部分之间使用复杂的交互方法.
其次,动态加载模块可以在操作系统运行时更改内核,这对内核的开发有好处,因为不用编译整个内核并重启系统.
这也对节约资源有好处,因为可以只按需要加载内核模块,减少内存占用.

\begin{qbox}{为什么叫“镜像”?}
	\lstinline{vmlinux} 被称为内核“镜像”(kernel image),是因为这个文件在加载操作系统时被用来在内存中创建内核的副本.
	详见Image一词的含义\cite{imageWik70:online}.
\end{qbox}

要了解Linux如何实现模块化的宏内核架构,我们在案例分析中重点介绍两方面的内容:内核模块是如何构建的、内核模块是如何加载的.

\subsection{内核构建系统}
\begin{readsrcbox}{Kernel Build System}
	内核的构建系统有关的文档在 \href{https://docs.kernel.org/kbuild/index.html}{\lstinline{Documentation/kbuild}} 目录.

	可以参考顶层目录的\lstinline{Makefile},各个目录下的\lstinline{Makefile}、\lstinline{Kbuild}.
	另外,\lstinline{script}目录下有用来链接 \lstinline{vmlinux}和有关构建模块的脚本.
\end{readsrcbox}
\begin{qbox}{如何找到某个子系统的有关文件和信息?}
	\lstinline{Maintainer} 文件中列出了Linux内核各个子系统的维护者,同时它还包括有关项目的网页和涉及的文件等有用的信息.
\end{qbox}
Linux内核的构建系统是一套基于GNU Make的递归式的构建系统,包括各个目录下的Makefile和为这些Makefile提供基础设施支持的其他文件,通常称为 \lstinline{kbuild}. \cite{LinuxKer71:online}

整个内核构建系统的目标主要就是内核镜像 \lstinline{vmlinux} 和可加载的模块.
各个Kbuild Makefile以向 \lstinline{obj-m} 和 \lstinline{obj-y} 变量增加内容的方式列出本目录应该构建的目标,
其中 \lstinline{obj-m} 变量中定义的目标会被构建成内核模块——后缀名为 \lstinline{.ko} 的一种ELF文件,而 \lstinline{obj-y} 变量中定义的目标会被 \lstinline{$(AR)}\footnote{一般就是GNU Binutils里面的ar,用于把多个 \lstinline{.o} 目标文件归档到一个文件中.} 合并到一个名为 \lstinline{built-in.a} 的库中,最后被链接进 \lstinline{vmlinux}.
顶层目录的Makefile递归地构建子目录.

有趣的是,内核的各个部分通常既可以编译到内核镜像中,也可以编译成可加载的内核模块,这是由目标是加入到 \lstinline{obj-m} 还是 \lstinline{obj-y} 来决定的.
例如Listing \ref{lst:obj-config}所示,当 \lstinline{CONFIG_BTRFS_FS}的值为y(Yes)时,BTRFS文件系统驱动——目标 \lstinline{btrfs.o}将作为内核内置的一部分编译;
而\lstinline{CONFIG_BTRFS_FS}的值为m(Module)时,它将编译成可加载的内核模块.

\begin{lstlisting}[language=make, caption=可配置的编译目标, label=lst:obj-config]
# fs/btrfs/Makefile
obj-$(CONFIG_BTRFS_FS) := btrfs.o
\end{lstlisting}


\section{进程管理}

\lstinline{struct task_struct} 是Linux内核的进程控制块(PCB).
它存储进程的标识符、进程的状态、指向存储进程上下文的数据结构的指针、调度所需的信息以及与进程相关的资源和指向其他 \lstinline{task_struct} 的指针等.
所有与进程有关的操作都会直接或间接地修改进程控制块来达到目的.
下面,我们介绍管理进程所需要的多种信息,并介绍它们是如何在 \lstinline{task_struct} 中表示的.

\begin{readsrcbox}{进程控制块}
	\lstinline{struct task_struct} 是在\lstinline{include/linux/sched.h}中定义的.
\end{readsrcbox}

首先是进程的状态,一个进程可能处于这些状态:
\begin{itemize}
	\item \lstinline{TASK_NEW}.
	      当一个进程刚创建时,它的状态就被设置为\lstinline{TASK_NEW},表示这个进程已经被创建,但是还没有开始运行.
	\item \lstinline{TASK_DEAD}.
	      进程已经结束.
	\item \lstinline{TASK_RUNNING}.
	      进程正在运行,即该进程在正在运行的进程的队列里.
	\item \lstinline{TASK_INTERRUPTIBLE} 或者 \lstinline{TASK_UNINTERRUPTIBLE}.
	      处于这两个状态的进程正在等待,但是只有 \lstinline{TASK_INTERRUPTIBLE} 状态的进程才会被信号唤醒.
	\item \lstinline{TASK_NOLOAD}.
	      类似 \lstinline{TASK_UNINTERUPTIBLE} 但是在统计数据中不被计入负载.
	      \footnote{\url{https://lore.kernel.org/lkml/alpine.LFD.2.11.1505112154420.1749@ja.home.ssi.bg/T/}}
	\item \lstinline{TASK_WAKING}.
	      进程已经被要求唤醒,但是还没有进入正在运行的进程队列.
	      \footnote{\url{https://lore.kernel.org/lkml/tip-e9c8431185d6c406887190519f6dbdd112641686@git.kernel.org/}}
	\item \lstinline{TASK_WAKEKILL}.
	      该进程收到SIGKILL信号时会被唤醒.
	\item \lstinline{TASK_TRACED}.
	      调试器暂停该进程来追踪它的运行状态.
\end{itemize}
这些状态被编码成掩码,读写进程的状态时,
需要用这些掩码操作 \lstinline{task_struct} 的 \lstinline{__state} 域.
比如识别一个进程是否处于某状态,
就要用它的 \lstinline{__state} 和 这个状态的掩码做与操作,
若结果为0则不处于这个状态.
这些状态中有的状态可以组合成新的掩码,方便使用.

进程控制块还需要存储有关进程上下文的信息,例如有关栈和堆的信息和CPU内部寄存器的值.
这一部分信息既和内核的内存管理有关又依赖于具体的硬件架构.
\begin{readsrcbox}{架构相关代码}
	为了提高可移植性,Linux内核的代码区分不依赖于具体硬件架构的代码和针对特定架构的代码.
	架构相关的代码全部在 \lstinline{arch} 目录下.
	例如 x86和x86\_64的代码位于 \lstinline{arch/x86}.

	显然,所有的汇编代码都应该放在该目录.
	内存管理和进程管理所需的某些功能也依赖于CPU的特定功能,
	需要根据硬件功能定义数据结构和执行具体的指令,
	这些定义和实现也位于 \lstinline{arch} 下具体架构的目录下.
	不同的架构的代码尽量暴露出相同的接口,供架构无关代码使用.

	某一架构上的具体实现的文档可以在 \href{https://docs.kernel.org/arch.html}{\lstinline{Document/arch.rst}} 中的列表找到.
	例如\href{https://docs.kernel.org/x86/kernel-stacks.html}{\lstinline{Document/x86/kernel-stacks.rst}}介绍了x86\_64 CPU上内核为每一个进程维护的若干个栈.
\end{readsrcbox}
\lstinline{task_struct}的定义中,状态后面最先出现的是一个 \lstinline{void *stack;} 域.
这不是该进程的用户态的栈的起始地址,
而是该进程的内核栈,每次该进程通过系统调用进入内核态时,
内核中的代码都在这个内核栈上执行.

Scheduling information is also included in the PCB, including priority of the process, schedule entity that represents the process and a pointer to the schedule class that this process is using.


\section{附录}
\tcblistof[\subsection]{question}{“小问题”列表}
\tcblistof[\subsection]{src}{“读源码”列表}

\printbibliography[heading=bibintoc]

\printindex

\newpage

\end{document}

%%% Local Variables:
%%% mode: latex
%%% TeX-master: t
%%% End:
