\section{设备管理}
\subsection{设备驱动模型}
计算机的CPU通过总线连接到设备,设备驱动程序通过总线与设备沟通,
并给其他部分暴露出友好的接口.
为总线、设备、驱动的相互操作而建立的程序框架在Linux内核中被称作设备驱动模型.
\cite{devicemodel}\cite{bovet2005understanding}

设备模型是由一系列相互联系的对象表示的.
表示总线、设备、驱动等的对象之间相互连接,构成层次关系,以便相互操作.
把它们连接起来的是 \lstinline{struct kobject},\index{k@\lstinline{kobject}}
一个嵌入其他结构体中提供功能的结构
(嵌入的机制与 \ref{process identifiers} 中的 \lstinline{list_head} 一样),
这其实是为了在C语言的简单结构下实现几个较复杂功能的一系列hack.

\lstinline{kobject} 接口提供的主要功能有\cite{kobject}:
\begin{itemize}
	\item 引用计数. 因为设备驱动模型对象之间相互引用的关系较复杂,
	      需要引用计数来确定对象的生命周期.
	\item 在需要释放对象时,根据对象的类型调用自定义的释放函数.
	\item 连接其他 \lstinline{kobject}.
	      设备和设备等之间需要构成父子关系.
	      不同对象之间还可以利用 \lstinline{kset} 或 \lstinline{klist} 构成collection.
	      collection 和父子关系构成了 \lstinline{kobject} 的层次结构.
	\item 向用户空间提供接口.
	      \lstinline{kobject} 在内核登记后就对应sysfs 虚拟文件系统
	      (Arch Linux下挂载在 \lstinline{/sys})下的一个目录.
	      用户态程序可以通过该虚拟文件系统的接口访问内核内部的对象.
	\item 事件的发送. 对象状态更改(如设备插拔)时可以向用户空间发送事件.
\end{itemize}
利用这些功能,内核就能构建一个灵活的设备驱动模型.

\subsubsection{总线}
总线由 \lstinline{struct bus_type} 表示.
\index{d@\lstinline{struct bus_type}}
所有设备在逻辑上都连在某一个总线上,所有驱动都要依靠总线访问设备.
因此总线的对象中的 \lstinline{struct subsys_private *p}
有两个装着 \lstinline{kobject} 的 \lstinline{kset},
分别引用该总线上的\emph{设备}和\emph{驱动}.

总线对象定义了总线普遍支持的操作,初始化具体的总线对象时要注册这些操作的函数指针,
其中一些操作是做一些总线相关的工作再调用设备的回调函数.
主要的操作有:
\begin{itemize}
	\item 在加入新的设备或驱动的时候调用的操作.
	      \begin{itemize}
		      \item 检查总线上的某个指定设备是否与指定驱动相匹配;
		      \item probe,调用驱动的probe来把设备加入到驱动的管理中;
		      \item ……
	      \end{itemize}
	\item 设备从总线上移除后的操作.
	\item 为设备准备DMA的操作.
	\item 管理设备电源、功耗的操作.
\end{itemize}

总线下的层次结构可以在sysfs 中看到.
每一种总线都在 \lstinline{/sys/bus/} 下有自己的目录.
目录下的 \lstinline{devices} 子目录是总线设备,
为指向全局的 \lstinline{devices} 子系统中的设备对象的符号链接;
\lstinline{drivers} 子目录下是总线的各个驱动.
例如 Listing~\ref{lst: pcie sysfs} 所示.

\begin{lstlisting}[caption={PCIE总线在sysfs中的结构}, label={lst: pcie sysfs}]
/sys/bus/pci-express
|-- devices
|   |-- 0000:00:07.0:pcie001 -> ../../../../devices/pci...
...
|   `-- 0000:00:1c.0:pcie010 -> ...
|-- drivers
|   |-- aer
|   |-- dpc
|   |-- pciehp
|   `-- pcie_pme
|-- drivers_autoprobe
|-- drivers_probe
`-- uevent
\end{lstlisting}

\subsubsection{设备}
设备在Linux设备驱动模型中的对象为 \lstinline{struct device}.
\index{d@\lstinline{struct device}}
其中存储的一些信息有:
\begin{itemize}
	\item 设备的名字;
	\item 设备的“父设备”,例如总线控制器、hub控制器等;
	\item 设备所在的总线类型;
	\item 设备使用的驱动;
\end{itemize}

\begin{qbox}{总线也是设备吗?}
	值得注意的是虽然已经有表示总线类型的 \lstinline{bus_type},
	总线也还是作为设备在设备驱动模型中表示,
	一种总线可能是另一种总线的子设备,
	例如挂在 PCI-E 总线下的 USB 总线.
\end{qbox}

\subsubsection{驱动}
设备驱动的对象是 \lstinline{struct device_driver}.
\index{d@\lstinline{struct device_driver}}
一个驱动对象只能存在一个实例,
也就是说驱动只有一份,一个驱动要管理与其关联的所有设备.
所以 \lstinline{struct device_driver} 的 \lstinline{struct driver_private *p}
要存储当前驱动所管理的所有设备.

向驱动添加设备的机制是存储在 \lstinline{struct device_driver} 中的函数指针
\lstinline{probe()}.\index{p@\lstinline{probe()}}
这个函数以 \lstinline{struct device} 为参数,应该做这样几件事\cite{devicedriver}:
\begin{itemize}
	\item 确定设备确实存在;
	\item 确定设备的型号和版本能被驱动支持;
	\item 为设备需要的数据结构分配内存空间;
	\item 初始化设备的硬件;
	\item 最后把设备驱动和设备绑定在一起,填上各自的引用.
\end{itemize}
\index{d@\lstinline{struct device}}
\begin{readsrcbox}{设备驱动模型}
	总线的表示 \lstinline{struct bus_type}\index{d@\lstinline{struct bus_type}}
	的定义在 \linuxsrc{include/linux/device/bus.h}.
	实际的总线定义自己的 \lstinline{struct bus_type},
	如PCI的 \lstinline{struct pci_bus_type} 在 \linuxsrc{drivers/pci/pci-driver.c}.

	设备的 \lstinline{struct device}\index{d@\lstinline{struct device}}
	定义在 \linuxsrc{include/linux/driver.h}.
	往往 \lstinline{struct device} 不单独使用,
	特定总线的设备会把 \lstinline{struct device} 嵌入在一个有更多信息的结构体中.
	比如PCI的 \lstinline{struct pci_dev} (\linuxsrc{include/linux/pci.h})、
	USB的 \lstinline{struct usb_dev} (\linuxsrc{include/linux/usb.h}).

	驱动的 \lstinline{struct device_driver} \index{d@\lstinline{struct device_driver}}
	在 \linuxsrc{include/linux/device/driver.h}.
	同样特定总线的驱动会把 \lstinline{struct device_driver} 嵌入到自己定义的结构体中.
\end{readsrcbox}

\subsection{设备驱动过程}
在 \ref{loading modules} 中,我们提到内核启动时,
bootloader加载vmlinux镜像,内核代码开始运行.
在开始启动init进程和运行init程序之间,
内核首先做最核心的初始化,这其中就包括设备驱动模型的初始化.
系统的总线等重要的设备的内核模块是静态连接到 vmlinux 内的,
在设备驱动模型初始化完成后,
这些模块的初始化程序会运行,注册对应的设备和总线类型,对总线进行初始化并设置中断.
这样基本的驱动框架就建立好了,总线的设备驱动就可以对总线上的设备进行搜索.
每当找到一个新的设备,首先初始化相关的对象,
然后通过总线的 \lstinline{match()} 和驱动的 \lstinline{probe()}
来遍历该总线下的所有驱动,如果发现匹配的驱动,则停止遍历,
把设备和驱动绑定起来,这样设备模型的总线、设备、驱动就都联系起来了.

现在再把用户空间的操作考虑进去.
嵌有 \lstinline{kobject} 的设备、驱动对象
在创建、删除的时候都会向用户空间传递 \lstinline{uevent}.
用户空间的程序可以通过监听事件,并且通过 sysfs 对设备驱动模型进行操作,
从而支持更灵活的设备使用方式,如热插拔.
在用户态受限的环境下管理设备的好处还有稳定性和安全性的提高.
同样在 \ref{loading modules} 中提过,
包括 Arch Linux 在内的大多数x86发行版都使用systemd的udev\index{udev}
系统来在用户态管理设备.
udev负责监听 \lstinline{uevent} 事件,
根据udev自带的和用户自定义的规则来做出对应的动作,
管理 \lstinline{/dev} 下的 device node 文件,
使其他程序可以通过 \lstinline{/dev} 下的文件与设备驱动程序交互.
udev规则一般可以实现这样的一些功能\cite{WritingUdev}:
\begin{itemize}
	\item 加载、卸载驱动所在的模块(见 \ref{loading modules}).
	\item 在 \lstinline{/dev} 下为设备设置友好的名字.
	\item 通知其他程序,传递设备的事件.
	\item 进行系统管理,如挂载刚刚插入的硬盘.
\end{itemize}

使用udev的命令行工具 \lstinline{udevadm} 可以直观地看到设备状态变化产生的操作.
比如运行 \lstinline{udevadm monitor --property} 命令,插入一个USB键盘,就可以看到udev根据规则和事件创建的device node.

\begin{lstlisting}
UDEV  [401515.997363] add      /devices/pci0 ... /usb3/.../event12 (input)
ACTION=add
DEVPATH=/devices/ ... /usb3/ ... /event12
SUBSYSTEM=input
DEVNAME=/dev/input/event12
\end{lstlisting}

\begin{notebox}
	这里描述的过程进行了大量的简化,许多重要的细节都忽略了.
	例如驱动可能在 \lstinline{probe()} 时还为准备好,
	这时它可以要求推迟 \lstinline{probe()} 的操作.
\end{notebox}

\begin{readsrcbox}{\lstinline{driver_init()}}
	初始化设备模型的函数入口为 \lstinline{driver_init()},
	位于 \linuxsrc{drivers/base/init.c},
	由 \linuxsrc{init/main.c} 中的 \lstinline{do_basic_setup()} 调用.
	\linuxsrc{drivers/base} 目录为内核驱动的“core”部分,
	负责硬件驱动模型内部的初始化和实现模型内部的功能等.

	\linuxsrc{drivers/base/core.c} 中的 \lstinline{device_add()}
	处理添加设备时的一系列操作,
	调用 \lstinline{bus_probe_device(dev);} 来匹配驱动.
	\lstinline{bus_probe_device()} 位于 \linuxsrc{drivers/base/dd.c},
	dd是“device/driver”的缩写,该文件中的函数处理设备和驱动之间的关系,
	例如绑定、通信等.
\end{readsrcbox}
%%% Local Variables:
%%% mode: latex
%%% TeX-master: "linux_zh"
%%% End:
